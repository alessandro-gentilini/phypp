\funcitem \cppinline|void phypp_check(bool b, ...)| \itt{phypp_check}

This function makes error checking easier. When called, it checks the value of \cppinline{b}. If \cppinline{b} is \cpptrue, then nothing happens. However if \cppinline{b} is \cppfalse, then the current file and line are printed to the standard output, followed by the other arguments of this function (they are supposed to compose an error message explaining what went wrong), and the program is immediately stopped. This function is used everywhere in the \phypp library to make sure that certain conditions are properly satisfied before doing a calculation, and it is essential to make the program crash in case something is unexpected (rather than letting it run hoping for the best, and often getting the worst).

Since this function is actually implemented by a preprocessor macro, you should not worry about its performance impact. It will only affect the performances when something goes wrong and the program is about to crash. However the calculation of \cppinline{b} can itself be costly (for example, you may want to check that a vector is sorted), and there is no way around this.

In addition, and if you have configured the \phypp library adequately, this function can print the ``backtrace'' of the program that lead to the error. This backtrace tells you which functions (or lines of code) the program was executing when the error occured. This information can be very useful to identify the source of the problem, but is only available if debugging informations are stored inside the compiled program. Note that this only affects the size of the executable on disk: debugging informations do not alter performances. If you need to solve complex problems, the backtrace may not be sufficient and you may need to use your favorite debugger.

\begin{example}
\begin{cppcode}
vec1i v;
// Suppose 'v' is read from the command line arguments.
v = read_from_somewhere_unsafe();

// The rest of the code needs at least 3 elements in the
// vector 'v', so we need to check that first.
phypp_check(v.size() >= 3, "this algorithm needs at least "
    "3 values in the input vector, but only ", v.size(),
    " were found");

// If we get past this point, we can proceed safely to use
// the first three elements
print(v[0]+v[1]+v[3]);
\end{cppcode}
\end{example}
