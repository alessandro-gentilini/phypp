\funcitem \cppinline|vec1u where(vec<D,bool>)| \itt{where}

This function will scan the \cppinline{bool} vector provided in argument, will store the flat indices (\ref{SEC:core:vec:indexing}) of each element which is \cpptrue, and will return all these indices in a vector. This is a very useful tool to filter and selectively modify vectors, and probably one of the most used function of the whole library.

\begin{example}
\begin{cppcode}
vec1i v = {4,8,6,7,5,2,3,9,0};
// We want to select all the elements which are greater than 3
// We use where() to get their indices
vec1u id = where(v > 3); // {0,1,2,3,4,7}
// Now we can check
v[id]; // {4,8,6,7,5,9}, good!

// The argument of where() can by as complex as you want
id = where(v < 3 || (v > 3 && v % 6 < 2)); // now guess

// It can also involve multiple vectors, as long as they have
// the same dimensions
vec1i w = {9,8,6,1,-2,0,8,5,1};
id = where(v > w || (v + w) % 5 == 0);
// The returned indices are valid for both v and w
v[id]; // {8,6,7,5,2,9}
w[id]; // {8,6,1,-2,0,5}
\end{cppcode}
\end{example}

\funcitem \cppinline|uint_t where_first(vec<D,bool>)| \itt{where_first}

\cppinline|uint_t where_last(vec<D,bool>)| \itt{where_last}

These functions will scan the \cppinline{bool} vector provided in argument, and return the flat index (\ref{SEC:core:vec:indexing}) of the first (or last) element which is \cpptrue, and \cppinline{npos} if all are \cppfalse.

\begin{example}
\begin{cppcode}
vec1i v = {4,8,6,7,5,2,3,9,0};
// We want to select the first element which is greater than 3
uint_t id;
id = where_first(v > 3); // 0
v[id];                   // 4
id = where_last(v > 3);  // 7
v[id];                   // 9
\end{cppcode}
\end{example}

\funcitem \cppinline|vec1u complement(vec v, vec1u id)| \itt{complement}

This function works in tandem with \cppinline{where}. Given a vector \cppinline{v} and a set of flat indices \cppinline{id} (\ref{SEC:core:vec:indexing}), it will return the complementary set of indices inside this vector, i.e., all the indices of \cppinline{v} that are \emph{not} present in \cppinline{id}.

\begin{example}
\begin{cppcode}
vec1i v = {1,5,6,3,7};
vec1u id = where(v > 4); // {1,2,4}
vec1u cid = complement(v, id); // {0,3}
\end{cppcode}
\end{example}

\funcitem \cppinline|uint_t lower_bound(T x, vec v)| \itt{lower_bound}

\cppinline|uint_t upper_bound(T x, vec v)| \itt{upper_bound}

\cppinline|array<uint_t,2> bounds(T x, vec v)| \itt{bounds}

\cppinline|array<uint_t,2> bounds(T x1, U x2, vec v)|

These functions use a binary search algorithm to locate the element in the input vector \cppinline{v} that is equal to or closest to the provided value \cppinline{x}. It is important to note that the binary search assumes that the elements in the input vector are \emph{sorted} by increasing value. This algorithm also assumes that the input vector does not contain any NaN value (\ref{SEC:support:math}).

The \cppinline{lower_bound} function locates the last element in \cppinline{v} that is less or equal to \cppinline{x}. If no such element is found, \cppinline{npos} is returned.

The \cppinline{upper_bound} function locates the first element in \cppinline{v} that is greater than \cppinline{x}. If no such element is found, \cppinline{npos} is returned.

The first \cppinline{bounds} function combines what both \cppinline{lower_bound} and \cppinline{upper_bound} do, and returns both indices in an array. The second \cppinline{bounds} function calls \cppinline{lower_bound} to look for \cppinline{x1}, and \cppinline{upper_bound} to look for \cppinline{x2}.

\begin{example}
\begin{cppcode}
vec1i v = {2,5,9,12,50};
bounds(0, v);   // {npos,0}
bounds(9, v);   // {2,3}
bounds(100, v); // {4,npos}
\end{cppcode}
\end{example}

\funcitem \cppinline|vec1u equal_range(T x, vec v)| \itt{equal_range}

Similarly to \cppinline{lower_bound}, \cppinline{upper_bound} and \cppinline{bounds}, this function uses a binary search algorithm to locate all the values in the input vector \cppinline{v} that are equal to \cppinline{x}. It then returns the flat indices of all these values in a vector. It is important to note that the binary search assumes that the elements in the input vector are \emph{sorted} by increasing value. This algorithm also assumes that the input vector does not contain any NaN value (\ref{SEC:support:math}).

If no such value is found, an empty vector is returned.

\begin{example}
\begin{cppcode}
vec1i v = {2,2,5,9,9,9,12,50};
equal_range(9, v); // {3,4,5}

// It's a faster version of
where(v == 9);
\end{cppcode}
\end{example}

\funcitem \cppinline|vec1u uniq(vec v)| \itt{uniq}

\cppinline|vec1u uniq(vec v, vec1u sid)|

This function will traverse the provided vector \cppinline{v} and find all the unique values. It will store the indices of these values (if a value is present more than once inside \cppinline{v}, the index of the first one will be used) and return them inside an index vector. The first version assumes that the values in \cppinline{v} are \emph{sorted} from the smallest to the largest. In the second version, \cppinline{v} may not be sorted, but the second argument \cppinline{id} contains indices that will sort \cppinline{v} (e.g., \cppinline{id} can be the return value of \cppinline{sort(v)}).

\begin{example}
\begin{cppcode}
// For a sorted vector
vec1i v = {1,1,2,5,5,6,9,9,10};
vec1u u = uniq(v); // {0,2,3,5,6,8}
v[u]; // {1,2,5,6,9,10} only unique values

// For an non-sorted vector
vec1i w = {5,6,7,8,6,5,4,1,2,5};
vec1u u = uniq(w, sort(w));
w[u]; // {1,2,4,5,6,7,8}
\end{cppcode}
\end{example}

\funcitem \vectorfunc \cppinline|bool is_any_of(T1 v1, vec<D2,T2> v2)| \itt{is_any_of}

This function looks if there is any value inside \cppinline{v2} that is equal to \cppinline{v1}. If so, it returns \cpptrue, else it returns \cppfalse.

\begin{example}
\begin{cppcode}
vec1i v = {7,4,2,1,6};
vec1i d = {5,6,7};
vec1b b = is_any_of(v, d); // {true, false, false, false, true}
\end{cppcode}
\end{example}

\funcitem \cppinline|void match(vec v1, v2, vec1u& id1, id2)| \itt{match}

This function traverses \cppinline{v1} and, for each value in \cppinline{v1}, looks for elements in \cppinline{v2} that have the same value. If one is found, the flat index of the element of \cppinline{v1} is added to \cppinline{id1}, and the flat index of the element of \cppinline{v2} is added to \cppinline{id2}. If other matches are found in \cppinline{v2} for this same value, they are ignored. Then the function goes on to the next value in \cppinline{v1}. The two vectors need not be the same size.

\begin{example}
\begin{cppcode}
vec1i v = {7,6,2,1,6};
vec1i w = {2,6,5,3};
vec1u id1, id2;
match(v, w, id1, id2);
id1; // {1,2,4}
id2; // {1,0,1}
v[id1] == w[id2]; // always true
\end{cppcode}
\end{example}

\funcitem \cppinline|bool astar_find(vec2b m, uint_t& x, y)| \itt{astar_find}

This function uses the $A^\star$ (``A star'') algorithm to look inside a 2D boolean map \cppinline{m} and, starting from the position \cppinline{x} and \cppinline{y} (i.e. \cppinline{m(x,y)}), find the closest point that has a value of \cpptrue. Once this position is found, its indices are stored inside \cppinline{x} and \cppinline{y}, and the function returns \cpptrue. If no element inside \cppinline{m} is \cpptrue, then the function returns \cppfalse.

\begin{example}
\begin{cppcode}
// Using '@' for true and '.' for false,
// assume we have the following boolean map,
// and that we start at the position indicated
// by 'S', the closest point whose coordinates
// will be returned by astar_find is indicated
// by an 'X'
vec2b m;
// .................
// .................
// .................
// ...@@@@@.........
// ...@@@@@.........
// ...@@@@@.........
// ...@@@@@.........
// ...@@@@X.........
// .............S...
// .................
// .................

uint_t x = 13, y = 2;
astar_find(m, x, y);
x; // 7
y; // 3
\end{cppcode}
\end{example}
