\funcitem \cppinline|vec1u mult_ids(vec v, uint_t i)| \itt{mult_ids}

\cppinline|vec1u mult_ids(array<uint_t> dims, uint_t i)|

This is only useful for multidimensional vectors. This function converts a ``flat'' index \cppinline{i} (\ref{SEC:core:vec:indexing}) into an array of multidimensional indices, following the dimensions of the provided vector \cppinline{v} (or, in the second version, following the dimensions \cppinline{dims}). The \cppinline{flat_id} function does the inverse job.

\begin{example}
\begin{cppcode}
vec2i v(2,3);
mult_ids(v,0); // {0,0}
mult_ids(v,1); // {0,1}
mult_ids(v,2); // {0,3}
mult_ids(v,3); // {1,0}
v[3] == v(1,0); // true
\end{cppcode}
\end{example}

\funcitem \cppinline|uint_t flat_id(vec v, ...)| \itt{flat_id}

This is only useful for multidimensional vectors. This function converts a group of multidimensional indices into a ``flat'' index (\ref{SEC:core:vec:indexing}), following the dimensions of the provided vector \cppinline{v}. The \cppinline{mult_ids} function does the inverse job.

\begin{example}
\begin{cppcode}
vec2i v(2,3);
flat_id(v,0,0); // 0
flat_id(v,0,1); // 1
flat_id(v,0,2); // 2
flat_id(v,1,0); // 3
v(1,0) == v[3]; // true
\end{cppcode}
\end{example}
