\funcitem \vectorfunc \cppinline|T e10(T)| \itt{e10}

\vectorfunc \cppinline|T sqr(T)| \itt{sqr}

\vectorfunc \cppinline|T invsqr(T)| \itt{invsqr}

These functions are just convenient shortcut for the \cppinline{pow()} function:
\begin{itemize}
\item \cppinline{e10(x)} computes the exponentiation in base $10$, i.e., $10^x$, and is equivalent to \cppinline{pow(10, x)},
\item \cppinline{sqr(x)} computes the square of its argument, i.e., $x^2$, and is equivalent to \cppinline{pow(x, 2)},
\item \cppinline{invsqr(x)} computes the inverse square of its argument, i.e., $1/x^2$, and is equivalent to \cppinline{pow(x, -2)}.
\end{itemize}

Because they are less generic, calling these functions instead of \cppinline{pow()} \emph{can} result in faster code, however any improvement should be marginal.

\funcitem \vectorfunc \cppinline|T clamp(T t, mi, ma)| \itt{clamp}

This function will compare its first argument against \cppinline{mi} and keep the largest value. Then, it will compare this latter value against \cppinline{ma} and return the smallest of both. In other words, the goal of this function is to make sure that a given value \cppinline{t} is bound to a specified interval \cppinline{[mi, ma]}. It is a more readable equivalent to \cppinline{min(max(t, mi), ma)}.

\begin{example}
\begin{cppcode}
vec1d x = {0.0, -0.05, 1.0, 0.5, 0.2, 1.06, 0.95, dnan};
// Make sure that the input values are between 0 and 1
clamp(x, 0.0, 1.0); // {0.0,0.0,1.0,0.5,0.2,1.0,0.95,nan}
\end{cppcode}
\end{example}

\funcitem \vectorfunc \cppinline|bool is_finite(T)| \itt{is_finite}

\vectorfunc \cppinline|bool is_nan(T)| \itt{is_nan}

The \cppinline{is_finite()} function will check if the provided argument is a finite number. In particular, for floating point numbers, it will return \cppinline{false} if the provided argument is either positive or negative infinity, or not a number (NaN). It will always return \cppinline{true} for integer numbers.

The \cppinline{is_nan()} function will only return \cppinline{true} if its argument is NaN, and is the safest way to isolate these values. Indeed, remember that NaN values have the special properties that \cppinline{x == fnan} will be \cppinline{false} even if the value of \cppinline{x} is NaN.

\begin{example}
\begin{cppcode}
vec1f x = {0.0, 1.0, finf, -finf, fnan};
is_finite(x); // {true,true,false,false,false}
is_nan(x);    // {false,false,false,false,true}
\end{cppcode}
\end{example}

\funcitem \vectorfunc \cppinline|int_t sign(T)| \itt{sign}

This function will return the sign of its argument. The most convenient way to achieve this is by returning an integer whose value is \cppinline{+1} for positive (or null) numbers, and \cppinline{-1} for strictly negative numbers. This allows convenient usage of this function inside more complex mathematical expressions without having to write conditional statements.

The result of calling this function on ``not a number'' values is undefined.

\begin{example}
\begin{cppcode}
// A stupid example.
vec1d x = {-1.0, 0.0, -1000.5, 20.0, 5.0};
// By computing the square of 'x', we loose its sign.
vec1d x2 = x*x;
// Say we want to get back to 'x'
// If we write: x2 = sqrt(x2)
// ... the value is right, but not the sign.
// Instead we can write:
x2 = sign(x)*sqrt(x2);
x2; // {-1.0,0.0,-1000.5,20.0,5.0}
\end{cppcode}
\end{example}
