\funcitem \cppinline|auto range(vec v)| \itt{range}

\cppinline|auto range(uint_t n)|

\cppinline|auto range(uint_t i0, uint_t n)|

This function returns a C++ \emph{range}, i.e., an object that can be used inside the C++ range-based \cppinline{for} loop. This range will generate integer values starting from \cppinline{0} (first and second version) or \cppinline{i0} (third version) to \cppinline{v.size()} (first version) or \cppinline{n} (second and third versions), that last value being \emph{excluded} from the range. This nice way of writing an integer \cppinline{for} loop actually runs as fast as (if not faster than) the classical way, and is less error prone.

\begin{example}
\begin{cppcode}
vec1i v = {4,5,6,8};

// First version
for (uint_t i : range(v)) {
    // 'i' goes from 0 to 3
    v[i] = ...;
}

// Note that the loop above generates
// *indices* inside the vector, while:
for (int i : v) { /* ... */ }
// ... generates *values* from the vector.

// Second version
for (uint_t i : range(3)) {
    // 'i' goes from 0 to 2
    v[i] = ...;
}

// Third version
for (uint_t i : range(1,3)) {
    // 'i' goes from 1 to 3
    v[i] = ...;
}
\end{cppcode}
\end{example}
