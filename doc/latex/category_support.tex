\texorhtml{\chapter{Support libraries \label{SEC:support}}}{\section{Introduction \label{SEC:support}}}

\texorhtml{In this chapter}{Here} we describe the set of helper functions that are part of the \phypp support library. These functions are not essential to the use of the \phypp library, but are mostly modular components that one may choose to use or not. All the support functions are sorted into broad categories to help you discover new functions and algorithm. Alternatively, if you know the name of a function and would like to read its documentation, an index is available at the end of this document.

Note that, in all this \texorhtml{section}{documentation}, the signature of the functions is given in pseudo-code, both for conciseness and readability. In particular, the following rules apply.

\begin{itemize}
\item The presence of the \vectorfuncsym symbol in front of the signature of the function means that this function is also available in a \emph{vectorized} form. This only applies to functions whose first argument is a scalar (i.e., not a vector). In this case, the vectorized form shares the same signature as the original form, but the first argument is promoted to a vector. Calling the vectorized version is the same as writing a loop to call the original version on each element of the vector. If the original function had a return value, the vectorized form returns a vector whose elements are the return value of each call, corresponding to each element of the input vector. The vectorized version can be faster than writing the loop manually.

\begin{example}
\begin{cppcode}
// Suppose this function is marked as vectorized
bool is_odd(uint_t)
// It means that there is another function with the
// same name, but that acts on a vector instead
vec<D,bool> is_odd(vec<D,uint_t>)

// It is used like this
vec1u v = {1,2,3,4,5};
vec1b b = is_odd(v);
// ... and is equivalent to
vec1b b(v.dims);
for (uint_t i : range(v)) {
    b[i] = is_odd(v[i]);
}
\end{cppcode}
\end{example}

\item If the function depends on an external library, the name of this library will be written before the signature, for example if LAPACK is needed you will find the symbol \lapacksym.

\item Template parameters are not declared explicitly. They are always written in uppercase, usually with a single character (e.g., \cppinline{T}, \cppinline{D}), or possibly two (e.g., a letter and a number), but never more. Letters \cppinline{T}, \cppinline{U}, \cppinline{V}, etc.~refer to template \emph{types}, while letters \cppinline{D}, \cppinline{N} or \cppinline{I} refer to template \emph{integers}.

\begin{example}
\begin{cppcode}
// Pseudo-code used in this section
void foo(T)
// Corresponding C++ code
template<typename T>
void foo(T);

// Pseudo-code used in this section
void foo(vec<D,T> v, U u)
// Corresponding C++ code
template<std::size_t D, typename T, typename U>
void foo(vec<D,T> v, U u);
\end{cppcode}
\end{example}

\item Template parameters are omitted when not relevant to the description of the function. In this case, it is implicitly assumed that the function will work for any type/value of these template parameters.

\begin{example}
\begin{cppcode}
// Pseudo-code used in this section
void sort(vec&)
// Corresponding C++ code
template<std::size_t D, typename T>
void sort(vec<D,T>&);
\end{cppcode}
\end{example}

\item There are only two kinds of arguments: input arguments, and input/output arguments. Input arguments are always spelled as plain types, e.g. \cppinline{T}, even if the actual signature of the function uses a constant reference, an r-value reference or a universal reference. The reason is that this implementation choice does not matter to the end user. What matters is the interface. The input/output parameters are always C++ references, e.g. \cppinline{T&}.

\begin{example}
\begin{cppcode}
// Pseudo-code used in this section
vec1u dims(vec)
// Corresponding C++ code could be either
template<std::size_t D, typename T>
vec1u dims(vec<D,T>);
// ... or
template<std::size_t D, typename T>
vec1u dims(const vec<D,T>&);
// The only difference is that the first version will
// always make a copy (or move) of its parameter, while
// the second may not. This optimization choice depends
// on the actual code inside the function, and has no
// consequence on how the function is actually used.
\end{cppcode}
\end{example}

\item The ellipsis \cppinline{...} is used to symbolize a list of multiple arguments whose length can vary depending on the context. These arguments are not spelled out explicitly, but the description of the function must make it clear what they are used for. Optionally, a type may be placed before the ellipsis to indicate that all the arguments must be of this same type.

\begin{example}
\begin{cppcode}
// Pseudo-code used in this section
uint_t flat_id(vec, ...)
// Corresponding C++ code
template<std::size_t D, typename T, typename ... Args>
uint_t flat_id(vec<D,T>, Args&& ...);
\end{cppcode}
\end{example}

\item The \cppinline{std} namespace is omitted for common standard types, like \cppinline{std::string} or \cppinline{std::array}.

\begin{example}
\begin{cppcode}
// Pseudo-code used in this section
uint_t length(string)
// Corresponding C++ code
uint_t length(const std::string&);
\end{cppcode}
\end{example}

\item For the particular case of \cppinline{std::array}, the individual elements inside the array can be named by placing the names inside curly braces after the type of the array.

\begin{example}
\begin{cppcode}
// Pseudo-code used in this section
uint_t distance(array {x,y})
// Corresponding C++ code
template<typename T>
uint_t length(const std::array<T,2>& a) {
    // x := a[0]
    // y := a[1]
}
\end{cppcode}
\end{example}

\item If the return type of a function in pseudo-code is \cppinline{auto}, it means that this return value is ``complex'' (usually a structure or a class) and it is not important to know its precise type. The description of the function must therefore make it clear how this return value can be used.

\end{itemize}

With this in mind, it is clear that this chapter focuses on the \emph{interface} that is provided by the library, rather than on the individual function themselves. In fact, a single interface may be composed of many different functions to take care of all the combination of types that the interface supports. If the reader is interested in all these overloads, or is experiencing a particular compiler error that cannot be easily fixed just by looking at the interface, then it is best to look directly into the code of the library. Although less readable than the pseudo-code used in this document, most of the time an effort is made to make the code as clear as possible. However, if a function is too hard to understand, I consider this as a bug that should be reported on the \texttt{github} issue tracker (seriously). Similarly, if you end up doing something wrong with the library, but that the compiler error message is too cryptic or too long, you may also fill in a bug report. Ensuring that clear error messages are sent to the user is a shared responsibility between compiler writers and library authors.
