\funcitem \cppinline|double derivate1_func(F f, double x, e)| \itt{derivate1_func}

\cppinline|double derivate1_func(F f, vec1d x, double e, uint_t i)|

\cppinline|double derivate2_func(F f, double x, e)| \itt{derivate2_func}

\cppinline|double derivate2_func(F f, vec1d x, double e, uint_t i)|

\cppinline|double derivate2_func(F f, vec1d x, double e, uint_t i1, i2)|

The \cppinline{derivate1_func} functions compute a numerical approximation of the first derivative of a function \cppinline{f} at point \cppinline{x}, with step \cppinline{e}. To do so, the algorithm uses a $5$ point symmetric sampling of the function around \cppinline{x}, ranging from \cppinline{(x - 2*e)} to \cppinline{(x + 2*e)}, and assumes that \cppinline{f} is continuous and can be well approximated by a $4^{\rm th}$ order polynomial within this range (the result will be exact if \cppinline{f} \emph{is} a polynomial of order less or equal to $4$).

The first version assumes that \cppinline{f} has a single scalar argument and returns ${\rm d}f/{\rm d}x$, while the second version assumes that \cppinline{f} has a single vector argument $x = \{x_i\}$, in which case the index \cppinline{i} is used to choose with respect to which of the vector's element the partial derivative should be taken, so the function returns $\partial f/\partial x_i$.

The \cppinline{derivate2_func} functions approximate instead the second derivative of \cppinline{f}. The first version assumes that \cppinline{f} has a single scalar argument and returns ${\rm d}^2f/{\rm d}x^2$, while the second version assumes that \cppinline{f} has a single vector argument $x = \{x_i\}$, in which case the index \cppinline{i} is used to choose with respect to which of the vector's element the partial second derivative should be taken, so the function returns $\partial^2 f/\partial {x_i}^2$. The third version computes the partial first derivatives with respect to elements \cppinline{i1} and \cppinline{i2}, so the function returns $\partial^2 f/(\partial x_{i1} \partial x_{i2})$.

The value of \cppinline{e} should be as small as possible, while preserving the fact that \cppinline{x+e} and \cppinline{x} are different numbers within the accuracy of \cppinline{double}, as well as \cppinline{f(x+e)} and \cppinline{f(x)} if the analytic form of \cppinline{f} would indeed predict that these are two different numbers.

\begin{example}
\begin{cppcode}
// A simple function
auto f1 = [](double x) {
    return cos(x*x);
};

// Estimate the derivative at x = pi
double df;
// The analytic value is
df = -2*dpi*sin(dpi*dpi);
df; // 2.70366
// First try with a poor precision
df = derivate1_func(f1, dpi, 1e-1);
df; // 2.70416, error of order 0.001
// And a little bit better
df = derivate1_func(f1, dpi, 1e-2);
df; // 2.70366, good enough

// We can also work with multiple arguments
auto f2 = [](vec1d x) {
    return cos(x[0])*sin(x[1]);
};

// Estimate the derivative with respect to x[0] at x = {pi, pi}
df = derivate1_func(f2, {dpi, dpi}, 1e-2, 0);
// Estimate the derivative with respect to x[1] at x = {pi, pi}
df = derivate1_func(f2, {dpi, dpi}, 1e-2, 1);
\end{cppcode}
\end{example}

\funcitem \cppinline|auto integrate_func(F f, double x1, x2, e = default)| \itt{integrate_func}

This function computes a numerical approximation of the integral of a function \cppinline{f} in the closed interval bounded by \cppinline{x1} and \cppinline{x2}, i.e., $\int_{x_1}^{x_2} {\rm d}x\,f(x)$. The algorithm uses the Simpson rule, iteratively doubling the degree $n$ (number of sampling points) until the result stabilizes within the chosen relative accuracy \cppinline{e} (which defaults to the machine epsilon matching the return type of \cppinline{f}). It is equivalent to the Numerical Recipe routine \cppinline{qsimp()}.

The function \cppinline{f} can return any arithmetic scalar type, including complex numbers. Vectors are not supported at the moment.

\begin{example}
\begin{cppcode}
// A simple function
auto gauss = [](double x) {
    return exp(-x*x);
};

double r = integrate_func(gauss, -10, 10);
// The analytic value is "sqrt(pi) x erf(10)" = 1,772453850905516...
r; // 1.77245385091... accurate indeed!
\end{cppcode}
\end{example}

\funcitem \cppinline|double integrate(vec<1,T> x, y)| \itt{integrate}

\cppinline|double integrate(vec<1,T> x, y, double x0, x1)|

\cppinline|double integrate(vec<2,T> x, vec<1,T> y)|

\cppinline|double integrate(vec<2,T> x, vec<1,T> y, double x0, x1)|

These functions perform the integral of a tabulated function $y(x)$, either covering the whole extent of the tabulated data, or only within a specified interval enclosed within \cppinline{x0} and \cppinline{x1}.

For the first two functions, the integration is performed using the trapezoid rule. Here, each value of \cppinline{x} corresponds to a given value of the function $y(x)$ in \cppinline{y}. The sampling can be arbitrary. The algorithm only assumes that \cppinline{x} is sorted and only contains unique values, and it also assumes that $y(x)$ is continuous (no check is performed for any of these hypotheses). The result will be exact if $y(x)$ is (piece-wise) linear. Note that if the analytical form of $y(x)$ is known, using \cppinline{integrate_func()} will usually be more efficient unless many integrals need to be computed for the same function.

The next two functions perform the integration of \emph{histogram} data by simply summing up the area of each ``bar'' of the histogram. In this case, \cppinline{x} is a two-dimensional vector such that \cppinline{x(0,_)} and \cppinline{x(1,_)} contain the lower and upper bound of each histogram point, respectively, while \cppinline{y} contains the average value of the function $y(x)$ between \cppinline{x(0,_)} and \cppinline{x(1,_)}. See the documentation of \cppinline{histogram()} for more detail about binned data. The values of \cppinline{x} need not be sorted, each interval can be of different size and the intervals can be disjoint. However, the algorithm assumes that the intervals do not overlap and that no interval is empty (no check is performed).

\begin{example}
\begin{cppcode}
vec1d x = rgen(-10.0, 10.0, 20); // build a grid of 20 points
vec1d y = exp(-x*x);             // sample a function

double r = integrate(x, y);
// The analytic value is "sqrt(pi) x erf(10)" = 1,772453850905516...
r; // 1.77197... not too bad

// Binned data example (see histogram())

// Generate a random data set
vec1d rnd = randomn(make_seed(42), 1000);

// Build some bins
vec2d xb = make_bins(-10, 10, 20);

// Compute the histogram of the random data
vec1d yb = histogram(rnd, xb);

// Now we can compute the integral of this binned data

// Note that the algorithm assumes that the 'y' parameter is the
// *average* of the function between 'x(0,_)' and 'x(1,_)', so we
// have to divide each value of 'yb' by the width of the
// corresponding bin
yb /= bin_width(xb);

// There is the integral
r = integrate(xb, yb);
r; // 1000 most probably
// The exact value here is simply "count(rnd >= -10 && rnd < 10)"
\end{cppcode}
\end{example}


\funcitem \cppinline|vec<1,T> cumul(vec<1,T> x, y)| \itt{cumul}

This function computes the cumulative integral of the tabulated function $y(x)$. The return value is a vector whose value \cppinline{v[i]} corresponds to the integral of $y(x)$ from \cppinline{x[0]} to \cppinline{x[i]}. The first value is therefore always zero, while the last value is equal to \cppinline{integrate(x,y)}. The function assumes that the vector \cppinline{x} is sorted and only contains unique values (no check is performed).

\begin{example}
\begin{cppcode}
vec1d x = rgen(-3.0, 3.0, 10); // build a grid of 10 points
vec1d y = exp(-x*x);           // sample a function

vec1d r = cumul(x, y);
// 'r' now holds something that is close to the formal integral of
// the Gaussian function, i.e., the error function, evaluated at each
// point 'x'.
\end{cppcode}
\end{example}
