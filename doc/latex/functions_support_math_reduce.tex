\funcitem \cppinline|vec<D-1,U> reduce(uint_t d, vec<D,T> v, F f)| \itt{reduce}

This is a meta function. It applies the function \cppinline{f} on slices of the vector \cppinline{v} along its \cppinline{d}th dimension (zero being the first dimension), stores the result of \cppinline{f} on each slice inside a new vector and returns it. The function \cppinline{f} must take a monodimensional vector in input and return a scalar value.

\begin{example}
\begin{cppcode}
// We create a 8 x 100 x 5 vector
vec3d v = dindgen(8, 100, 5);

// We want to compute the mean of each set of 100 values.
// In other words:
vec2d m(8,5);
for (uint_t i : range(8))
for (uint_t j : range(5)) {
    m(i,j) = mean(v(i,_,j));
}

// The function reduce() does exactly that, with some optimizations
m = reduce(1, v, [](vec1d x) { return mean(x); });
//         ^
// The '1' above indicates that we want to reduce the second
// dimension
\end{cppcode}
\end{example}

Note that, since the above example can be a common operation, the same can also be achieved with \cppinline{m = partial_mean(1, v)}. Most of the common reduction functions also have a ``\cppinline{partial_...}'' version.

\funcitem \cppinline|void run_dim(uint_t d, vec..., F f)| \itt{run_dim}

\begin{advanced}
This is a meta function. It applies the function \cppinline{f} on slices of the \cppinline{n} vectors (the \cppinline{vec...} in the signature) along their \cppinline{d}th dimension. These \cppinline{n} vectors must have exactly the same size and dimensions. The function \cppinline{f} must take \cppinline{n} monodimensional vectors in input. Any return value will be ignored.

This is a more generic version of \cppinline{reduce()}. Contrary to \cppinline{reduce()}, \cppinline{run_dim()} does not return anything. If you want to use the data to create and fill one or several vectors, you have to do this yourself by capturing the output vectors into the lambda function (see the example below). This gives more freedom, for example to output multiple vectors. On top of this, \cppinline{run_dim()} allows you to operate on multiple input vectors at the same time.

\begin{example}
\begin{cppcode}
// We create two 8 x 100 x 5 vectors
vec2d v = dindgen(100, 5);
vec2d w = dindgen(100, 5);

// For each set of 100 values in both vectors, we want to
// compute the average of values in v weighted by the values in w,
// as well as store the sum of the weights.
// In other words:
vec1d m(8,5);
vec1d s(8,5);
for (uint_t i : range(8))
for (uint_t j : range(5)) {
    // First sum the weights
    s(i,j) = total(w(i,_,j));
    // Then compute the average, normalized by the sum
    m(i,j) = total(v(i,_,j)*w(i,_,j))/s(i,j);
}

// The function run_dim() can be used to write this without an
// explicit loop, and will likely result in faster code
run_dim(0, v, w, [&](uint_t i, vec1d xv, vec1d xw) {
    s[i] = total(xw)
    m[i] = total(xv*xw)/s[i];
});
\end{cppcode}
\end{example}
\end{advanced}

\funcitem \cppinline|U total(vec<D,T> v)| \itt{total}

\cppinline|vec<D-1,U> partial_total(uint_t d, vec<D,T> v)| \itt{partial_total}

The function \cppinline{total()} returns the sum of all the values inside \cppinline{v}. If the vector contains integers or booleans, the sum will also be an integer (note however that it is preferable to use \cppinline{count()} for boolean vectors). In all other cases, the sum will be a double precision floating point number.

The function \cppinline{partial_total()} will apply \cppinline{total()} on the \cppinline{d}th dimension of the vector (zero being the first dimension) and reduce its number of dimensions by one.

\begin{example}
\begin{cppcode}
vec1d v = {1,2,3,4,5};
total(v); // 15

vec2d w = {{1,2,3,4,5}, {10,20,30,40,50}};
partial_total(0, w); // {11,22,33,44,55}
\end{cppcode}
\end{example}

\funcitem \cppinline|uint_t count(vec<D,bool> v)| \itt{count}

\cppinline|vec<D-1,uint_t> partial_count(uint_t d, vec<D,bool> v)| \itt{partial_count}

The function \cppinline{count()} returns the number of values in \cppinline{v} that are \cppinline{true}.

The function \cppinline{partial_count()} will apply \cppinline{count()} on the \cppinline{d}th dimension of the vector (zero being the first dimension) and reduce its number of dimensions by one.

\begin{example}
\begin{cppcode}
vec1d v = {false,false,true,true,false};
count(v); // 2

vec2d w = {{false,true,true,false}, {false,false,true,false}};
partial_count(0, w); // {0,1,2,0}
\end{cppcode}
\end{example}

\funcitem \cppinline|double fraction_of(vec<D,bool> v)| \itt{fraction_of}

\cppinline|vec<D-1,double> partial_fraction_of(uint_t d, vec<D,bool> v)| \itt{partial_fraction_of}

The function \cppinline{fraction_of()} returns the number of values in \cppinline{v} that are \cppinline{true} divided by the total number of values.

Calling \cppinline{fraction_of()} on an empty vector returns not-a-number.

The function \cppinline{partial_fraction_of()} will apply \cppinline{fraction_of()} on the \cppinline{d}th dimension of the vector (zero being the first dimension) and reduce its number of dimensions by one.

\begin{example}
\begin{cppcode}
vec1d v = {false,false,true,true,false};
fraction_of(v); // 0.4

vec2d w = {{false,true,true,false}, {false,false,true,false}};
fraction_of(0, w); // {0,0.5,1,0}
\end{cppcode}
\end{example}

\funcitem \cppinline|double mean(vec v)| \itt{mean}

\cppinline|vec<D-1,double> partial_mean(uint_t d, vec<D,T> v)| \itt{partial_mean}

The function \cppinline{mean()} computes the average of the values in \cppinline{v}, i.e., the sum divided by the number of elements.

Calling \cppinline{mean()} on an empty vector returns not-a-number.

The function \cppinline{partial_mean()} will apply \cppinline{mean()} on the \cppinline{d}th dimension of the vector (zero being the first dimension) and reduce its number of dimensions by one.

\begin{example}
\begin{cppcode}
vec1d v = {-1,1,0.5,2,1.5};
mean(v); // 0.8

vec2d w = {{0,1,2,3,4}, {0,0,1,3,6}};
partial_mean(0, w); // {0,0.5,1.5,3,5}
\end{cppcode}
\end{example}

\funcitem \cppinline|T median(vec<D,T> v)| \itt{median}

\cppinline|T inplace_median(vec<D,T>& v)| \itt{inplace_median}

\cppinline|vec<D-1,T> partial_median(uint_t d, vec<D,T> v)| \itt{partial_median}

The function \cppinline{median()} computes the median of the values in \cppinline{v}, i.e., it returns the element of \cppinline{v} that, if \cppinline{v} was sorted, would be located at the position \cppinline{n/2} (rounded down), with \cppinline{n} being the number of elements in \cppinline{v}.

Because not-a-number values cannot be ordered, they are simply ignored in the computation. Also,
contrary to \cppinline{mean()}, calling \cppinline{median()} on an empty vector will trigger an error, since \cppinline{median()} can only return a value from \cppinline{v}.

The function \cppinline{inplace_median()} will return exactly the same value as \cppinline{median()}. However, the algorithm that is used to compute the median needs to re-order the elements inside the input vector. With \cppinline{median()}, it is necessary to do a full copy of \cppinline{v} to prevent it from being modified, and this can decrease the performances. \cppinline{inplace_median()} will not do this copy, and will therefore modify \cppinline{v}. If you can accept this, then use this function since it will always be faster. Note however that the order of the values in \cppinline{v} resulting from a call to \cppinline{inplace_median()} is unspecified.

The function \cppinline{partial_median()} will apply \cppinline{median()} on the \cppinline{d}th dimension of the vector (zero being the first dimension) and reduce its number of dimensions by one.

\begin{example}
\begin{cppcode}
vec1d v = {-1,1,0.5,2,1.5};
median(v); // 1

inplace_median(v); // also 1
// However, now 'v' still contains the same values, but there is
// no way to know in which order
v; // could be {-1,0.5,2,1,1.5}

vec1u x = {1,2};
median(x); // 2 by convention

vec2d w = {{0,1,2,3,4}, {0,0,1,3,6}, {-1,2,3,3,5}};
partial_median(0, w); // {0,1,2,3,5}
\end{cppcode}
\end{example}

\funcitem \cppinline|T percentile(vec<D,T> v, double p)| \itt{percentile}

\cppinline|vec<D-1,T> partial_percentile(uint_t d, vec<D,T> v, double p)| \itt{partial_percentile}

\cppinline|vec<1,T> percentiles(vec<D,T> v, double ...)| \itt{percentiles}

The function \cppinline{percentile()} computes the \cppinline{p}th percentile of the values in \cppinline{v}, i.e., it returns the element of \cppinline{v} that, if \cppinline{v} was sorted, would be located at the position \cppinline{p*n} (rounded down), with \cppinline{n} being the number of elements in \cppinline{v}.

Because not-a-number values cannot be ordered, they are simply ignored in the computation. Also,
contrary to \cppinline{mean()}, calling \cppinline{percentile()} on an empty vector will trigger an error, since \cppinline{percentile()} can only return a value from \cppinline{v}.

The function \cppinline{percentiles()} computes multiple percentiles at the same time. This can be faster than calling \cppinline{percentile()} repeatedly.

The function \cppinline{partial_percentile()} will apply \cppinline{percentile()} on the \cppinline{d}th dimension of the vector (zero being the first dimension) and reduce its number of dimensions by one.

\begin{example}
\begin{cppcode}
vec1d v = {-1,1,0.5,2,1.5};
percentile(v, 0.0);  // -1 (this is the minimum)
percentile(v, 0.25); // 0.5
percentile(v, 0.5);  // 1 (this is the median)
percentile(v, 0.75); // 1.5
percentile(v, 1.0);  // 2 (this is the maximum)

percentiles(v, 0.5, 0.25, 0.75); // {1,0.5,1.5}
\end{cppcode}
\end{example}

\funcitem \cppinline|double rms(vec<D,T> v)| \itt{rms}

\cppinline|vec<D-1,double> partial_rms(uint_t d, vec<D,T> v)| \itt{partial_rms}

The function \cppinline{rms()} computes the square root of the average of the square of the values in \cppinline{v}, i.e., \cppinline{sqrt(mean(sqr(v)))}. This is often called the ``root mean square'', or RMS.

Calling \cppinline{rms()} on an empty vector returns not-a-number.

The function \cppinline{partial_rms()} will apply \cppinline{rms()} on the \cppinline{d}th dimension of the vector (zero being the first dimension) and reduce its number of dimensions by one.

\begin{example}
\begin{cppcode}
vec1d v = {-1,1,0.5,2,1.5};
rms(v); // 1.30384
\end{cppcode}
\end{example}

\funcitem \cppinline|double stddev(vec<D,T> v)| \itt{stddev}

\cppinline|vec<D-1,double> partial_stddev(uint_t d, vec<D,T> v)| \itt{partial_stddev}

The function \cppinline{stddev()} computes the standard deviation of the values in \cppinline{v}, i.e., \cppinline{sqrt(mean(sqr(v - mean(v))))}.

Calling \cppinline{stddev()} on an empty vector returns not-a-number.

The function \cppinline{partial_stddev()} will apply \cppinline{stddev()} on the \cppinline{d}th dimension of the vector (zero being the first dimension) and reduce its number of dimensions by one.

\begin{example}
\begin{cppcode}
vec1d v = {-1,1,0.5,2,1.5};
stddev(v); // 1.02956
\end{cppcode}
\end{example}

\funcitem \cppinline|T mad(vec<D,T> v)| \itt{mad}

\cppinline|vec<D-1,T> partial_mad(uint_t d, vec<D,T> v)| \itt{partial_mad}

The function \cppinline{mad()} computes the median absolute deviation of the values in \cppinline{v}, i.e., \cppinline{median(fabs(v - median(v)))}. Note that, for a set of values with a distribution close to a Gaussian, the median absolute deviation is just the standard deviation divided by $\sim1.48$. Like the median compared to the mean, it is however less sensitive to strong outliers.

Calling \cppinline{mad()} on an empty vector will trigger an error.

The function \cppinline{partial_mad()} will apply \cppinline{mad()} on the \cppinline{d}th dimension of the vector (zero being the first dimension) and reduce its number of dimensions by one.

\begin{example}
\begin{cppcode}
vec1d v = {-1,1,0.5,2,1.5};
mad(v); // 0.5
\end{cppcode}
\end{example}

\funcitem \cppinline|T min(vec<D,T> v)| \itt{min}

\cppinline|T min(vec<D,T> v, uint_t& i)|

\cppinline|vec<D-1,T> partial_min(uint_t d, vec<D,T> v)| \itt{partial_min}

\cppinline|T max(vec<D,T> v)| \itt{max}

\cppinline|T max(vec<D,T> v, uint_t& i)|

\cppinline|vec<D-1,T> partial_max(uint_t d, vec<D,T> v)| \itt{partial_max}

The first \cppinline{min()} and \cppinline{max()} functions return respectively the minimum and maximum value of the vector \cppinline{v()}. The second \cppinline{min()} and \cppinline{max()} functions do the same, but also store the index of this value in the output parameter \cppinline{i}.

Calling \cppinline{min()} or \cppinline{max()} on an empty vector will trigger an error.

The functions \cppinline{partial_min()} and \cppinline{partial_max()} will apply respectively \cppinline{min()} and \cppinline{max()} on the \cppinline{d}th dimension of the vector (zero being the first dimension) and reduce its number of dimensions by one.

\begin{example}
\begin{cppcode}
vec1d v = {-1,1,0.5,2,1.5};
min(v); // -1
max(v); // 2

uint_t id;
min(v, id); // -1
id;         // 0
max(v, id); // 2
id;         // 3

vec2d w = {{0,1,2,3,4}, {0,0,1,3,6}, {-1,2,3,3,5}};
partial_min(0, w); // {-1,0,1,3,4}
partial_max(0, w); // {0,2,3,3,6}
\end{cppcode}
\end{example}

\funcitem \cppinline|uint_t min_id(vec v)| \itt{min_id}

\cppinline|uint_t max_id(vec v)| \itt{max_id}

These functions return respectively the index of the minimum and maximum values in \cppinline{v}. Calling them on an empty vector will trigger an error.

\begin{example}
\begin{cppcode}
vec1d v = {-1,1,0.5,2,1.5};
min_id(v); // 0
max_id(v); // 3
\end{cppcode}
\end{example}

\funcitem \cppinline|vec1u histogram(vec v, vec<2,V> b)| \itt{histogram}

\cppinline|vec<1,W> histogram(vec<D,T> v, vec<D,W> w, vec<2,U> b)|

The first \cppinline{histogram()} function computes the histogram of the values inside the vector \cppinline{v} using the bins defined in \cppinline{b} (see \cppinline{make_bins()}). In other words, it counts the number of values of \cppinline{v} that fall in each bin of \cppinline{b}, and returns a new vector containing these counts.

The second \cppinline{histogram()} function produces a weighed histogram, where each value in \cppinline{v} comes with a weight as given in \cppinline{w}. The weights of all the values that fall within a given bin are summed and stored inside the returned vector. The first function is equivalent to the second function with all the weights equal to one.

\begin{example}
\begin{cppcode}
// First generate some values
auto seed = make_seed(42);
vec1d v = randomn(seed, 1000);

// Then build 7 bins covering -5 to +5
vec2d b = make_bins(-5, 5, 7);

// Compute the histogram
vec1u c = histogram(v, b);
c; // {0, 18, 216, 506, 237, 23, 0}
\end{cppcode}
\end{example}

\funcitem \cppinline|vec2u histogram2d(vec x, y, vec<2,U> bx, by)| \itt{histogram2d}

\cppinline|void histogram2d(vec x, y, vec<2,U> bx, by, F func)|

The first \cppinline{histogram2d()} function computes the two-dimensional histogram of the values inside the vectors \cppinline{x} and \cppinline{y} using the bins defined in \cppinline{bx} and \cppinline{by} respectively (see \cppinline{make_bins()}). In other words, it counts the number of values of \cppinline{x} that fall in each bin of \cppinline{bx}, then within this bin it counts the number of values of \cppinline{y} that fall in each bin of \cppinline{by}, and returns a new 2D vector containing these counts.

\begin{example}
\begin{cppcode}
// First generate some values
auto seed = make_seed(42);
vec1d x = randomn(seed, 1000);
vec1d y = randomn(seed, 1000) + x;

// Then build 7 bins covering -6 to +6
// For simplicity we will be using the same bins for both 'x' and 'y'
// but it is of course possible to use something different
vec2d b = make_bins(-6, 6, 7);

// Compute the histogram
vec2u c = histogram2d(x, y, b, b);
c;
//   0   0   0   0   0   0   0
//   0   3   2   0   0   0   0
//   1   25  105 57  7   0   0
//   0   7   114 316 138 4   0
//   0   0   3   61  124 25  0
//   0   0   0   0   4   4   0
//   0   0   0   0   0   0   0
\end{cppcode}
\end{example}

The second \cppinline{histogram2d()} function is more generic and simply bins the values without counting them. For each bin \cppinline{i} and \cppinline{j} of \cppinline{bx} and \cppinline{by}, respectively, it produces a vector containing the indices of the values of \cppinline{x} and \cppinline{y} that fall in these bins, and gives \cppinline{i}, \cppinline{j} and this vector to the function \cppinline{func}.

\begin{example}
\begin{cppcode}
// Using the same setup as the example above.
// This time we also have an extra variable.
vec1d z = randomn(seed, 1000);

// Instead of simply computing the histogram of 'x' and 'y',
// we want to compute the average value of 'z' in each bin.
// We will store this into
vec2d mz(7,7);

// Here we go, we bin in 'x' and 'y'
histogram2d(x, y, b, b, [&](uint_t i, uint_t j, vec1u ids) {
    // We are in the 'x' bin 'i' and 'y' bin 'j'.
    // The function tells us that the values 'ids' fall inside
    // this bin.

    // Now we just have to compute the average of 'z' for these
    // elements
    mz(i,j) = mean(z[ids]);
});
\end{cppcode}
\end{example}

\funcitem \cppinline|vec<D,bool> sigma_clip(vec<D,T> v, double x)| \itt{sigma_clip}

This function computes \cppinline{sigma = 1.48*mad(v)} (see \cppinline{mad()}) and returns a vector containing \cppinline{true} for the values that are within \cppinline{-x*sigma} and \cppinline{+x*sigma} of the median of \cppinline{v}, and \cppinline{false} otherwise.

It can be used to identify and flag out strong outliers from a data set.

\begin{example}
\begin{cppcode}
// We generate some "normal" values
auto seed = make_seed(42);
vec1d z = randomn(seed, 1000);
// But we add by hand some strong outliers
z[200] = 1e6;
z[600] = -1e9;

// Look for 10 sigma outliers
vec1b cl = sigma_clip(z, 10);
// Very likely:
cl[200]; // false
cl[600]; // false
// ... and all the others should be true
\end{cppcode}
\end{example}
