\funcitem \cppinline|vec1u sort(vec)| \itt{sort}

\cppinline|void inplace_sort(vec&)| \itt{inplace_sort}

These functions will change the order of the elements inside a given vector so that they are sorted from the smallest to the largest. The difference between the two versions is that \cppinline{sort} does not actually modify the provided vector, but rather returns a vector containing indices inside the provided vector, and \cppinline{inplace_sort} directly modifies the provided vector. The later is the fastest of the two, but it is less powerful.

\begin{example}
\begin{cppcode}
// First version
vec1i v = {1,5,6,3,7};
vec1u id = sort(v); // {0,3,1,2,4}
v[id]; // {1,3,5,6,7} is sorted
// now, id can also be used to modify the order of
// another vector of the same dimensions

// Second version
inplace_sort(v);
v; // {1,3,5,6,7} is sorted
\end{cppcode}
\end{example}

\funcitem \cppinline|bool is_sorted(vec)| \itt{is_sorted}

This function just traverses the whole input vector and checks if its elements are sorted by increasing value.

\begin{example}
\begin{cppcode}
// First version
vec1i v = {1,5,6,3,7};
is_sorted(v); // false
inplace_sort(v);
v; // {1,3,5,6,7}
is_sorted(v); // true
\end{cppcode}
\end{example}

\funcitem \cppinline|vec<1,T> reverse(vec<1,T>)| \itt{reverse}

This function will inverse the order of all the elements inside the provided vector.

\begin{example}
\begin{cppcode}
vec1i v = {1,2,3,4,5,6};
vec1i w = reverse(v); // {6,5,4,3,2,1}
\end{cppcode}
\end{example}

\funcitem \cppinline|vec<1,T> shift(vec<1,T> v, int_t n, T d = 0)| \itt{shift}

This function will shift the position of the elements inside the provided vector \cppinline{v} by a given amount of indices \cppinline{n}. Elements that would go outside of the bounds of the vector are destroyed. New elements are inserted and default constructed, or assigned the default value \cppinline{d} (optional argument).

\begin{example}
\begin{cppcode}
vec1i v = {1,2,3,4,5};
vec1i sr = shift(v, 2);
sr; // {0,0,1,2,3}
vec1i sl = shift(v, -2, 99);
sl; // {3,4,5,99,99};
\end{cppcode}
\end{example}
