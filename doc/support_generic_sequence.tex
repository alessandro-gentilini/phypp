\item \cppinline|vec<D,int_t>  indgen(...)| \itt{indgen}

\cppinline|vec<D,uint_t> uindgen(...)| \itt{uindgen}

\cppinline|vec<D,float>  findgen(...)| \itt{findgen}

\cppinline|vec<D,double> dindgen(...)| \itt{dindgen}

These functions will create a new vector, whose dimensions are specified in argument (like in the dimension constructor, \ref{SEC:core:vec:constructor}, or \cppinline{vec::resize()}, \ref{SEC:core:vec:member_fun}). After this vector is created, the function will fill it with values that start at \cppinline{0} and increment by steps of \cppinline{1} until the end of the vector.

\begin{example}
\begin{cppcode}
vec1i v = indgen(5);    // {0,1,2,3,4}
vec2u w = uindgen(3,2); // {{0,1}, {2,3}, {4,5}}
\end{cppcode}
\end{example}
